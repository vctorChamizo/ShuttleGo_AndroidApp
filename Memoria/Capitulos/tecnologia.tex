\chapter{Tecnolog�a empleada}

\begin{resumen}
%resumen
\end{resumen}

\section{Introducci�n}
%-------------------------------------------------------------------
\label{cap1:sec:introduccion}
En este cap�tulo se explicar�n las diferentes tecnolog�as usadas para llevar a cabo este proyecto.
Mostraremos qu� servicios externos hemos necesitado y c�mo hemos usado sus APIs para integrarlos en nuestra aplicaci�n.

\section{Cliente}
\label{cap1:sec:cliente}


\section{Servidor}
\label{cap1:sec:Servidor}
El servidor se encuentra alojado en Firebase, el cual permite el desarrollo del proyecto en Node.js mediante varios paquetes.
Para desplegar el c�digo del servidor en Firebase utilizamos el paquete firebase-functions, mientras que para el almacenamiento persistente de datos tenemos el paquete firebase-admin.
Mediante este paquete, Firebase nos ofrece un servicio, Cloud Firestore, el cual nos permite salvar los datos en una base de datos NoSQL.
%-------------------------------------------------------------------
\section*{\NotasBibliograficas}
%-------------------------------------------------------------------
\TocNotasBibliograficas

Citamos algo para que aparezca en la bibliograf�a\ldots
\citep{ldesc2e}

\medskip

Y tambi�n ponemos el acr�nimo \ac{CVS} para que no cruja.

Ten en cuenta que si no quieres acr�nimos (o no quieres que te falle la compilaci�n en ``release'' mientras no tengas ninguno) basta con que no definas la constante \verb+\acronimosEnRelease+ (en \texttt{config.tex}).


%-------------------------------------------------------------------
\section*{\ProximoCapitulo}
%-------------------------------------------------------------------
\TocProximoCapitulo

...

% Variable local para emacs, para  que encuentre el fichero maestro de
% compilaci�n y funcionen mejor algunas teclas r�pidas de AucTeX
%%%
%%% Local Variables:
%%% mode: latex
%%% TeX-master: "../Tesis.tex"
%%% End:
