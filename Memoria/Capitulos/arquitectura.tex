\chapter{Arquitectura}

\section{Introducci�n}
%-------------------------------------------------------------------
\label{cap1:sec:introduccion}
En este apartado se explicar� la arquitectura de toda la aplicaci�n en detalle.
En la primera secci�n se explicar� la arquitectura general de toda la aplicaci�n, mientras que en los siguientes apartados se describir� de forma m�s detallada la arquitectura del cliente y del servidor. 

\section{Arquitectura general de la aplicaci�n}
\label{cap1:sec:arquitecturacliente}
La arquitectura de este proyecto sigue el patr�n t�pico de cliente-servidor, donde el cliente se encarga de la interacci�n del sistema con el usuario, mientras que la funci�n del servidor consiste en asegurarse que se cumplan todas las reglas de negocio y de la interacci�n con la base de datos.
Adem�s, en este caso, el cliente es tambi�n el encargado de hacer de intermediario con la API que gestiona los mapas, esto es debido a que queremos evitar sobrecargar el servidor con demasiadas peticiones y aunque hubieramos querido, el plan gratuito de Firebase no permite conectar el servidor con servicios externos a Google.
Con todo esto, la aplicaci�n tendr�a la siguiente estructura:
\section{Arquitectura del cliente}
\label{cap1:sec:arquitecturacliente}


\section{Arquitectura del servidor}
\label{cap1:sec:arquitecturaservidor}
El servidor tiene una arquitectura multicapa formada por tres partes bien delimitadas:
\begin{itemize}  
    \item Controlador frontal: Representado por el archivo "index.js". Es el encargado de recibir las peticiones. Adem�s tambi�n realiza peque�as validaciones de datos que est�n estrechamente ligados al formato de las peticiones y que podr�an suponer un error de ejecuci�n (ej: comprobar que no lleguen los datos vacios) y validaciones de autorizaci�n (ej: comprobar que el usuario que ha mandado la petici�n de eliminar un origen es un administrador);
    \item Capa de negocio: Formada por todos los servicios de aplicaci�n ubicados en la carpeta "business". El objetivo de esta capa es asegurarse de que se cumplen las diferentes reglas de negocio (ej: No se pueden a�adir m�s pasajeros a un viaje si est�n todas las plazas ocupadas).
    \item Capa de integraci�n: Formada por los "Data Access Objects" (DAOs) de la carpeta "dataAccess". Esta capa se encarga de dar acceso al sistema a la base de datos mediante operaciones simples.
\end{itemize}
Una vez que el cliente ha enviado una petici�n al servidor se realizan los clientes pasos:
\begin{enumerate}
    \item Se ejecuta la funci�n del controlador asociada a la petici�n, donde primeramente se realizan comprobaciones de autorizaci�n y validaci�n de datos.
    \item Esa funci�n llama a un servicio de aplicaci�n.
    \item El servicio de aplicaci�n lleva a cabo las reglas de negocio llamando a las funciones de los DAOs.
    \item Los DAOs interactuan directamente con la base de datos obteniendo, introduciendo y modificando informaci�n.
\end{enumerate}
%-------------------------------------------------------------------
\section*{\NotasBibliograficas}
%-------------------------------------------------------------------
\TocNotasBibliograficas

Citamos algo para que aparezca en la bibliograf�a\ldots
\citep{ldesc2e}

\medskip

Y tambi�n ponemos el acr�nimo \ac{CVS} para que no cruja.

Ten en cuenta que si no quieres acr�nimos (o no quieres que te falle la compilaci�n en ``release'' mientras no tengas ninguno) basta con que no definas la constante \verb+\acronimosEnRelease+ (en \texttt{config.tex}).


%-------------------------------------------------------------------
\section*{\ProximoCapitulo}
%-------------------------------------------------------------------
\TocProximoCapitulo

...

% Variable local para emacs, para  que encuentre el fichero maestro de
% compilaci�n y funcionen mejor algunas teclas r�pidas de AucTeX
%%%
%%% Local Variables:
%%% mode: latex
%%% TeX-master: "../Tesis.tex"
%%% End:
